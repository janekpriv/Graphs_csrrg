\documentclass[a4paper,12pt]{article}
\usepackage[utf8]{inputenc}
\usepackage{graphicx}
\usepackage{hyperref}
\usepackage{amsmath, amssymb}
\usepackage{geometry}
\geometry{margin=1in}
\usepackage{listings}
\usepackage{xcolor}

\lstdefinestyle{mystyle}{
    backgroundcolor=\color{gray!10}, % Lekko szare tło
    commentstyle=\color{green},      % Kolor komentarzy
    keywordstyle=\color{blue},       % Kolor słów kluczowych
    stringstyle=\color{red},         % Kolor stringów
    basicstyle=\ttfamily\footnotesize, % Podstawowy styl tekstu
    breaklines=true,                 % Łamanie linii
    numbers=left,                    % Numeracja linii
    numberstyle=\tiny\color{gray},   % Styl numeracji linii
    frame=single,                     % Ramka wokół kodu
}
\lstset{style=mystyle}

\title{Dokumentacja funkcjonalna}
\author{Jan Michorek, Anastasiia Prodius}
\date{\today}

\begin{document}

\maketitle

\tableofcontents

\section{Wstęp}
Nasza aplikacja pozwala użytkownikowi wczytywać grafy z pliku 
o rozszerzeniu csrrg, a następnie dzielić graf na dowolną liczbę częśći.

\section{Technologie}
Aplikacja została zbudowana przy użyciu następujących technologii:
\begin{itemize}
    \item Język programowania: C
    \item Biblioteki: To be defined 

\end{itemize}

\section{Architektura systemu}
Aplikacja składa się z modułów:
\begin{itemize}
    \item Moduł generowania grafów
    \item Moduł analizy grafów
    \item Moduł wczytywania grafów z pliku
\end{itemize}

\section{Struktura kodu}
Kod źródłowy podzielony jest na następujące pliki i katalogi:
\begin{verbatim}
- src/
    - graph.c
    - csr_parser.c
- lib/
    - graph.h
    - csr_parser.h
- output/
\end{verbatim}


\section{Przykładowy kod}
Do przechowywania grafów, służą tak zdefiniowane struktury:
\begin{lstlisting}[language=C, caption=struktura grafu]
typedef struct Node {
    int id;
    int ne;
    struct Node **links;
} *Node;

typedef struct Graph {
    GraphType type;
    int n;  
    Node *nodes;
} Graph;
\end{lstlisting}
W bezpośredniej generacji grafów uczestniczą funkcję podane poniżej:
\begin{lstlisting}[language=C, caption=generacja]
void link_nodes(Node, Node);
Node create_Node(int);
Graph * graph_init(int, GraphType type);
\end{lstlisting}
Zczytywanie z pliku o rosrzerzeniu csrrg oraz zczytywanie z macierzy.
\begin{lstlisting}[language=C, caption=parsowanie i konwertowanie]
void parse_csrrg(FILE *in);
void parse_mat(FILE *in);
\end{lstlisting}

\section{Podsumowanie}
Aplikacja w podstawowej wersji zapewni użytkownikowi obsługę plików crrsg, i konwertowania sposobów zapisu grafów.

\end{document}
